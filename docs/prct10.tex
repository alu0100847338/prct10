\documentclass[spanish,a4paper,10pt]{article}

\usepackage{latexsym,amsfonts,amssymb,amstext,amsthm,float,amsmath}
\usepackage[spanish]{babel}
\usepackage[latin1]{inputenc}
\usepackage[dvips]{epsfig}
\usepackage{doc}

\begin{document}
\title{N�mero $\pi$}
\author{T�cnicas Experimentales \\ Pr�ctica de Laboratorio \#10}
\date{11 de abril de 2014}

\maketitle

\begin{abstract}
El objetivo de esta pr�ctica es entregar un documento en PDF mediante \textsf{Latex}
cuyo tema principal sea el n�mero $\pi$. En dicho documento se deber�n mostrar sus partes m�s importantes.
\end{abstract}

\section{Historia del n�mero $\pi$}
Desde que el ser humano desarroll� la capacidad de contar y empez� a 
explorar las propiedades de esos entes abstractos llamados n�meros se 
ha sentido fascinado por lo que generacones de mentes curiosas iban descubriendo. 
A medida que nuestro conocimiento sobre ellos aumentaba, alguno de ellos llamaban 
especialmente la atenci�ny, a veces, hasta los mitificabamos. Pero si hay un n�mero 
que ha fascinado y que ha hecho correr r�os de tinta, ese es $\pi$ (pi). Un n�mero que, 
a pesar de contar con una larga historia, no fue "bautizado" con el nombre con el que le
conocemos hoy hasta el siglo XVIII.

Pi es el n�mero que se obtiene de dividir la longitud de una circunferencia por su di�metro.
No importa el tama�o de la circunferencia. Grande o peque�a, la proporci�n
entre su longitud y su di�metro es siempre la misma. Aunque es probable que esta propiedad 
fuera conocidad con anterioridad,los grandes estudiosos de este n�mero fueron los
antiguos griegos, como Anax�goras, Hip�crates de Quios o Antifronte de Atenas. Anteriormente, 
el valor de $\pi$ se determinaba, casi con seguridad, mediante medidas
experimentales. Arqu�medes fue el primero que sepamos, que realiz� una estimaci�n te�rica de su valor. 
Usando pol�gonos circunscritos e inscritos, determin� que $\pi$
era mayor que 223/71 y menor que 22/7.

Siguiendo el m�todo de Arqu�medes, otros matem�ticos consiguieron mejores aproximaciones, 
y ya en 480, Zu Chongzhi hab�a determinado el valor de $\pi$ entre 3.1415926 y 
3.1415927.

\section{C�lculo del n�mero $\pi$ y posible ejemplo de su c�lculo aproximado}
$\pi$ se puede calcular mediante integraci�n:
$$\int_{0}^{1} \! \frac{4}{1+x^2}\, dx = 4(atan(1) -atan(0) = \pi $$

Esta integral se puede aproximar num�ricamente con una d�rmula de cuadratura.

Si se utiliza la regla del punto medio se obtien:

\begin{center}
 $ \pi \approx \frac{1}{n} \sum\limits_{i=1}^{n}f(x_i)\,$,
 con $f(x) = \frac{4}{1+x^2}\,$,
 $x_i = \frac{i - \frac{1}{2}}{n}$,
 para $i = 1, \dots, n$
\end{center}


Ahora, como ejemplo, podr�amos crear un programa en \textsf{Python} que reciba 
como entrada el n�mero de subintervalos con los que se desea abordar la 
aproximaci�n de $\pi$.

A partir de �l se deber� calcular y mostrar por la consola:
\begin{enumerate}
 \item 
   Los extremos de los sbintervalos.
 \item 
   El punto $x_i$.
 \item
   El valor de la funci�n de aproximaci�n de $pi$, $f(x_i)$.
 \item
   El resultado de la aproximaci�n.
 \item  La constante $pi$ con treinta y cinco decimales.
\end{enumerate}

Por ejemplo, si se utilizan 4 subintervalos, la salida deber�a ser:

\begin{footnotesize}
\begin{verbatim}
Introduzca el n�mero de intervalos (n > 0): 4
Subintervalo: [0 , 0.25] x_i: 0.125 fx_i: 3.93846
Subintervalo: [0.25, 0.5 ] x_i: 0.375 fx_i: 3.50685
Subintervalo: [0.5 , 0.75] x_i: 0.625 fx_i: 2.8764
Subintervalo: [0.75, 1 ] x_i: 0.875 fx_i: 2.26549

El valor aproximado de PI es: 3.14680051839
El valor de PI con 35 decimales: 3.1415926535897931159979634685441852 \end{verbatim}
\end{footnotesize}




